\documentclass[english,man]{apa6}

\usepackage{amssymb,amsmath}
\usepackage{ifxetex,ifluatex}
\usepackage{fixltx2e} % provides \textsubscript
\ifnum 0\ifxetex 1\fi\ifluatex 1\fi=0 % if pdftex
  \usepackage[T1]{fontenc}
  \usepackage[utf8]{inputenc}
\else % if luatex or xelatex
  \ifxetex
    \usepackage{mathspec}
    \usepackage{xltxtra,xunicode}
  \else
    \usepackage{fontspec}
  \fi
  \defaultfontfeatures{Mapping=tex-text,Scale=MatchLowercase}
  \newcommand{\euro}{€}
\fi
% use upquote if available, for straight quotes in verbatim environments
\IfFileExists{upquote.sty}{\usepackage{upquote}}{}
% use microtype if available
\IfFileExists{microtype.sty}{\usepackage{microtype}}{}

% Table formatting
\usepackage{longtable, booktabs}
\usepackage{lscape}
% \usepackage[counterclockwise]{rotating}   % Landscape page setup for large tables
\usepackage{multirow}		% Table styling
\usepackage{tabularx}		% Control Column width
\usepackage[flushleft]{threeparttable}	% Allows for three part tables with a specified notes section
\usepackage{threeparttablex}            % Lets threeparttable work with longtable

% Create new environments so endfloat can handle them
% \newenvironment{ltable}
%   {\begin{landscape}\begin{center}\begin{threeparttable}}
%   {\end{threeparttable}\end{center}\end{landscape}}

\newenvironment{lltable}
  {\begin{landscape}\begin{center}\begin{ThreePartTable}}
  {\end{ThreePartTable}\end{center}\end{landscape}}

  \usepackage{ifthen} % Only add declarations when endfloat package is loaded
  \ifthenelse{\equal{\string man}{\string man}}{%
   \DeclareDelayedFloatFlavor{ThreePartTable}{table} % Make endfloat play with longtable
   % \DeclareDelayedFloatFlavor{ltable}{table} % Make endfloat play with lscape
   \DeclareDelayedFloatFlavor{lltable}{table} % Make endfloat play with lscape & longtable
  }{}%



% The following enables adjusting longtable caption width to table width
% Solution found at http://golatex.de/longtable-mit-caption-so-breit-wie-die-tabelle-t15767.html
\makeatletter
\newcommand\LastLTentrywidth{1em}
\newlength\longtablewidth
\setlength{\longtablewidth}{1in}
\newcommand\getlongtablewidth{%
 \begingroup
  \ifcsname LT@\roman{LT@tables}\endcsname
  \global\longtablewidth=0pt
  \renewcommand\LT@entry[2]{\global\advance\longtablewidth by ##2\relax\gdef\LastLTentrywidth{##2}}%
  \@nameuse{LT@\roman{LT@tables}}%
  \fi
\endgroup}


\ifxetex
  \usepackage[setpagesize=false, % page size defined by xetex
              unicode=false, % unicode breaks when used with xetex
              xetex]{hyperref}
\else
  \usepackage[unicode=true]{hyperref}
\fi
\hypersetup{breaklinks=true,
            pdfauthor={},
            pdftitle={An Extension of the QWERTY Effect: Not Just the Right Hand, Expertise and Typability Predict Valence Ratings of Words},
            colorlinks=true,
            citecolor=blue,
            urlcolor=blue,
            linkcolor=black,
            pdfborder={0 0 0}}
\urlstyle{same}  % don't use monospace font for urls

\setlength{\parindent}{0pt}
%\setlength{\parskip}{0pt plus 0pt minus 0pt}

\setlength{\emergencystretch}{3em}  % prevent overfull lines

\ifxetex
  \usepackage{polyglossia}
  \setmainlanguage{}
\else
  \usepackage[english]{babel}
\fi

% Manuscript styling
\captionsetup{font=singlespacing,justification=justified}
\usepackage{csquotes}
\usepackage{upgreek}

 % Line numbering
  \usepackage{lineno}
  \linenumbers


\usepackage{tikz} % Variable definition to generate author note

% fix for \tightlist problem in pandoc 1.14
\providecommand{\tightlist}{%
  \setlength{\itemsep}{0pt}\setlength{\parskip}{0pt}}

% Essential manuscript parts
  \title{An Extension of the QWERTY Effect: Not Just the Right Hand, Expertise
and Typability Predict Valence Ratings of Words}

  \shorttitle{QWERTY EFFECT EXTENSION}


  \author{Erin M. Buchanan\textsuperscript{1}~\& Kathrene D. Valentine\textsuperscript{2}}

  % \def\affdep{{"", ""}}%
  % \def\affcity{{"", ""}}%

  \affiliation{
    \vspace{0.5cm}
          \textsuperscript{1} Missouri State University\\
          \textsuperscript{2} University of Missouri  }

  \authornote{
    Erin M. Buchanan is an Associate Professor of Quantitative Psychology at
    Missouri State University. K. D. Valentine is a Ph.D.~candidate at the
    University of Missouri.
    
    Correspondence concerning this article should be addressed to Erin M.
    Buchanan, 901 S. National Ave, Springfield, MO 65897. E-mail:
    \href{mailto:erinbuchanan@missouristate.edu}{\nolinkurl{erinbuchanan@missouristate.edu}}
  }


  \abstract{Typing is a ubiquitous daily action for many individuals; yet, research
on how these actions have changed our perception of language is limited.
The QWERTY effect is an increase in valence ratings for words typed more
with the right hand on a traditional keyboard (Jasmin \& Casasanto,
2012). Although this finding is intuitively appealing given both right
handed dominance and the smaller number of letters typed with the right
hand, extension and replication of the right side advantage is
warranted. The present paper reexamined the QWERTY effect within the
embodied cognition framework (Barsalou, 1999) and found that the right
side advantage is replicable to new valence stimuli, as well as
experimental manipulation. Further, when examining expertise, right side
advantage interacted with typing speed and typability (i.e., alternating
hand keypresses or finger switches) portraying that both skill and our
procedural actions play a role in judgment of valence on words.}
  \keywords{keyboard, valence, QWERTY, word norms \\

    
  }





\usepackage{amsthm}
\newtheorem{theorem}{Theorem}
\newtheorem{lemma}{Lemma}
\theoremstyle{definition}
\newtheorem{definition}{Definition}
\newtheorem{corollary}{Corollary}
\newtheorem{proposition}{Proposition}
\theoremstyle{definition}
\newtheorem{example}{Example}
\theoremstyle{definition}
\newtheorem{exercise}{Exercise}
\theoremstyle{remark}
\newtheorem*{remark}{Remark}
\newtheorem*{solution}{Solution}
\begin{document}

\maketitle

\setcounter{secnumdepth}{0}



From its creation in 1868, to its appearance in our homes today, the
QWERTY keyboard has held the interest of psychologists. The process of
typing on a keyboard requires many procedures to function in tandem,
which creates a wealth of actions to research (Inhoff \& Gordon, 1997).
Rumelhart and Norman (1982)'s computer model of skilled typing is still
highly influential. They hypothesize that typing results from the
activation of three levels of cognition: the word level, the keypress
level, and the response level. They believe that after word perception,
the word level is activated, causing the keypress level to initiate a
schema of the letters involved in typing the word. This schema includes
the optimal position on the keyboard for that specific hand-finger
combination to move to at the appropriate time for individual
keystrokes. Concurrently, the response system sends feedback information
to initiate a keypress motion when the finger is in the appropriate
space. Their theory proposes that schemata and motion activations occur
simultaneously, constantly pulling or pushing the hands and fingers in
the right direction.

While many studies have focused on errors in typing to investigate
response system feedback (Logan, 1999), Logan (2003) argued for parallel
activation of keypresses. He examined the Simon effect to show more than
one letter is activated at the same time, and consequently, the second
keypress motion is begun before the first keypress is done. The Simon
effect occurs when congruent stimuli create faster responses than
incongruent stimuli, much like the Stroop task (Simon, 1990; Simon \&
Small, 1969). For example, if we are asked to type the letter f (a left
handed letter), we type it faster if the f is presented on the left side
of the screen. Similarly, Rieger (2004) reported finger-congruency
effects by altering a Stroop task: participants were required to respond
to centrally presented letters based on color-key combinations. When the
letter and color were congruent (i.e.,a right-handed letter was
presented in the designated color for a right response), the skilled
typists' responses were faster than incongruent combinations. Further,
this effect was present when participants responded to items with their
hands crossed on the responding device, suggesting the effect was
expertise-based rather than experiment-response based. These results
imply that automatic actions stimulate motor and imagery representations
concurrently and may be linked together in the brain (Hommel, Müsseler,
Aschersleben, \& Prinz, 2001; Logan \& Zbrodoff, 1998; Rieger, 2004).
This dual activation of motor and imagined items is the basis for
embodied cognition, a rapidly expanding field in psychology (Barsalou,
1999; Salthouse, 1986).

\subsection{Embodied Cognition}\label{embodied-cognition}

While the mind was traditionally considered an abstract symbol processor
(Newell \& Simon, 1976), newer cognitive psychology theories focus on
the interaction between the brain's sensorimotor systems and mental
representations of events and objects (Barsalou, 1999; Zwaan, 1999). The
interplay between these systems has been found in both neurological
(Hauk, Johnsrude, \& Pulvermüller, 2004; Lyons et al., 2010; Tettamanti
et al., 2005) and behavioral research (Cartmill, Goldin-Meadow, \&
Beilock, 2012; Holt2006; Zwaan \& Taylor, 2006). Motor representations
of tasks are activated even when not specifically asked to perform the
task, and if the action is well-learned, the task is perceived as
pleasant (Beilock \& Holt, 2007; Ping, Dhillon, \& Beilock, 2009; Yang,
Gallo, \& Beilock, 2009). For example, Beilock and Holt (2007) asked
novice and expert typists to pick which one of two letter dyads they
preferred, which were either different hand combinations (CJ) or same
finger combinations (FV). They found that novices have no preference in
selection, while expert typists more reliably picked the combinations
that were easier to type. To show that this effect was due to covert
motor representation activation, and thus, expanding on findings from
Van den Bergh, Vrana, and Eelen (1990), participants also made
preference selections while repeating a keypress combination. When
expert motor planning was distracted by remembering the pattern
presented, no preference for letter dyads was found, indicating that the
simultaneous activation of the motor representation was necessary to
influence their likability ratings. Similar embodied findings have also
been portrayed with emotionally charged sentences and facial movements
(Havas, Glenberg, \& Rinck, 2007), positive-negative actions, such as
head nodding or arm movements (Glenberg, Webster, Mouilso, Havas, \&
Lindeman, 2009; Ping et al., 2009), and perceptuomotor fluency
(Oppenheimer, 2008; Yang et al., 2009).

\subsection{Body Specificity
Hypothesis}\label{body-specificity-hypothesis}

Using an embodied framework, Casasanto (2009) has proposed that
handedness dictates preference because our representations of actions
are grounded in our physical interactions with the environment. In
several studies, he portrayed that handedness influenced preference for
spatial presentation (i.e.,left handed individuals associate
\enquote{good} with left, while right handed individuals associate
\enquote{good} with right), which in turn influenced judgments of
happiness and intelligence and our decision making in hiring job
candidates and shopping. In all these studies, participants reliably
selected the hand-dominant side more often, which does not match
cultural or neurolinguistic representations of positive-is-right and
negative-is-left (Davidson, 1992). These findings imply that our
handedness is a motor expertise that causes ease of action on the
dominant side to positively influence our perceptions of items presented
on that side. Further, Casasanto (2011) compiled a review of body
specific actions and their representation in the brain using fMRIs.
Handedness interacted with imagining actions, reading action, and
perceiving the meanings of action verbs, such that fMRI patterns were
mirrored for left and right handed participants matching their dominant
side.

\subsection{The QWERTY Effect}\label{the-qwerty-effect}

These effects inspired Jasmin and Casasanto (2012) to propose the idea
that typing, an action that often replaces speaking, has the ability to
create semantic changes in how we perceive words. The asymmetrical
arrangement of letters on the QWERTY keyboard increases fluency of
typing letters on the right side because there are fewer keys, and thus,
less competition for fingers. That arrangement should then cause us to
perceive the letters on the right side as more positive and letters on
the left side as more negative. Consequently, words that are composed of
more letters from the right side (the right side advantage; RSA) should
be rated as more positive than those with more letters on the left. They
found this preference for RSA over three languages (English, Spanish,
and Dutch), and the effect was even stronger on words created after the
invention of the QWERTY keyboard (i.e.,lol), as well as evident in
pseudowords such as plook. However, in contrast to the body specificity
hypothesis, left and right handed participants showed the same trend in
effects for positive-is-right words.

\section{Current Study}\label{current-study}

The current study examined the right side advantage's interaction with
traditional embodied cognition definitions (expertise, fluency). We
analyzed the different implications of the body specificity hypothesis
and a more general embodied hypothesis by testing the following: 1) To
examine embodied cognition, we coded each word for number of hand
alternations (akin to Beilock and Holt (2007)'s different hand
preferences). Given that typing involves the procedural action system,
we would also expect to find that increased hand switches are positively
related to ratings of valence because words that are typed on
alternating hands are easier to type. 2) The interaction between RSA and
switches was examined to determine if these hypotheses can be combined
(i.e.,we only like right handed words because we have to switch back and
forth to type the more commonly used letters, such as \emph{e} or
\emph{a}).

\section{Experiment 1}\label{experiment-1}

\section{Method}\label{method}

\subsection{Participants}\label{participants}

Participants (\emph{N} = 546) were recruited from the university
undergraduate human subject pool and received course credit for their
time. 65233 rows of data were present for these participants, where 504
participants included complete data (i.e.,120 rows, see below), 39 were
missing one data point, and 3 were missing many data points. All data
points were included, and missing data points were usually computer
error (i.e., freezing during the experiment) or participant error (i.e.,
missed key press).

Rating data were screened for multivariate outliers, and one
participant's ratings were found to have extreme Mahalanobis distance
scores ({\textbf{???}}) but were kept in the data set. 11.5 percent of
the sample was left-handed, 0.2 marked ambigdextrious, and 0.4 was
missing handedness information. The average typing speed was 48.20
(\emph{SD} = 13.45, and the average percent accuracy rate for the typing
test was 92.59 (\emph{SD} = 8.63).

\subsection{Materials}\label{materials}

Both Experiment 1 and Experiment 2 use the English ANEW ({\textbf{???}})
norms to create stimuli for this study, in an effort to replicate Jasmin
and Casasanto (2012) experiments, and 2743 words were selected for this
experiment. Pseudowords were selected from Appendix E of the
supplementary materials presented from the QWERTY publication. These
words were coded as described below for RSA, switches, word length, and
letter frequency. Average word length was 4.85 (\emph{SD} = 1.51; range
= 3 - 13). All materials, data, and the \emph{R}markdown document that
created this manuscript are avaliable at our Open Science Foundation
(OSF) page: \url{https://osf.io/zs2qj/}.

\subsection{Coding}\label{coding}

Each of the words used in this experiment and Experiment 2 were coded
for control and experimental variables. Control variables included word
length and average letter frequency. Average letter frequency was
created by averaging the English letter frequency ({\textbf{???}}) for
each letter in a word. Words with high average letter frequencies
contain more commonly used letters (\emph{e, t, a, o}); while words with
lower frequencies use more of the less common letters (\emph{z, q, x,
j}). Experimental variables included RSA, number of hand switches, and
number of finger switches. Typing manuals were consulted, and letters
were coded as left (\emph{q, w, e, r, t, a, s, d, f, g, z, x, c, v, b})
or right-handed letters (\emph{y, u, i, o, p, h, j, k, l, n, m}). Left
handed letters were coded with -1 and right handed letters with +1,
which created summed scores indicating the overall right side advantage
for a word. Words were coded for the number of hand switches within a
word using the left-right coding system described above. Finally, the
number of finger switches were coded using traditional typing manuals
for each finger. Finger switches was highly correlated with word length,
\emph{r} = .89, and therefore, word length was excluded as a control
variable due the interest in typing skill for experimental hypotheses.

\subsection{Procedure}\label{procedure}

Upon consent to participate in the experiment, participants were given a
typing test by using a free typing test website ({\textbf{???}}). Each
participant typed Aesop's Fables for one minute before the website would
reveal their typing speed and accuracy rate, which was recorded by the
experimenter. After this test, participants indicated their dominant
writing hand. Participants were then given 120 of the possible stimuli
to rate for pleasantness (60 real words, 60 pseudowords). This smaller
number of stimuli was used to control fatigue/boredom on participants.
These stimuli were counterbalanced across participants, and the order of
the stimuli was randomized. Participants were told to rate each word for
how pleasant it seemed using a 9 point Likert type scale (1 - \emph{very
unpleasant}, 4 - \emph{neutral}, 9 - \emph{very pleasant}). The same
self-assessment manikin from Jasmin and Casasanto (2012) was shown to
participants at the top of the computer screen to indicate the points on
the Likert scale. The words appeared in the middle of the screen in 18
point Arial font. Participants then typed the number of their rating on
the computer keyboard. Once they rated all stimuli, participants were
debriefed and allowed to leave.

\section{Results}\label{results}

\subsection{Data Analytic Plan}\label{data-analytic-plan}

Because each participant constituted multiple data points within the
dataset, a multilevel model was used to control for correlated error
({\textbf{???}}). ({\textbf{???}})'s \emph{nlme} package in \emph{R} was
used to calculate these analyses. A maximum likelihood multilevel model
was used to examine hypotheses of interactions between typing speed,
hand/finger switching, and RSA while adjusting for letter frequency when
predicting item pleasantness ratings. Pseudowords and real words were
examined separately in two multilevel model analyses. Participants were
included as a random intercept factor, as comparison to a non-random
intercept was significant (see Table \ref{tab:model-table1}). Typing
speed, finger/hand switches, and RSA were mean centered before analyses
to control for multicollinearity.

\subsection{Main Effects}\label{main-effects}

After setting participants as a random intercept factor, letter
frequency was used as an adjustor variable. As seen in Table
\ref{tab:model-table1}, this variable was not a significant predictor
for pseudowords, \emph{b} = -0.01, but was a significant predictor for
real words, \emph{b} = 0.05. All predictor statistics are provided in an
Excel sheet on the OSF page for each step of the model. Next, the main
effects of typing speed, hand switches, finger switches, and RSA were
added to the models for pseudowords and real words. In both models, the
addition of these variables overall was significant, \emph{p}
\textless{} .001. For psuedowords, typing speed was not a significant
predictor of valence ratings, \emph{b} = 0.00, \emph{t}(541) = 1.45,
\emph{p} = .148. Similarly, typing speed was not a significant predictor
for valence ratings on real words, \emph{b} = 0.00, \emph{t}(544) =
-0.22, \emph{p} = .826. In contrast, the measures of typability in hand
and finger switching were significant for both pseudowords and real
words. For pseudowords, increased hand switching, \emph{b} = -0.03,
\emph{t}(31995) = -3.35, \emph{p} = .001, and increased finger
switching, \emph{b} = -0.06, \emph{t}(31995) = -4.78, \emph{p}
\textless{} .001, decreased the overall valence ratings. However
increased hand switching, \emph{b} = 0.04, \emph{t}(32141) = 2.77,
\emph{p} = .006, increased valence ratings for real words, while
increased finger switching, \emph{b} = -0.08, \emph{t}(32141) = -6.29,
\emph{p} \textless{} .001, decreased the overall valence ratings. Even
adjusting for these typing style variables, the RSA effect replicated
for both pseudowords, \emph{b} = 0.05, \emph{t}(31995) = 10.63, \emph{p}
\textless{} .001, and real words, \emph{b} = 0.05, \emph{t}(32141) =
7.39, \emph{p} \textless{} .001. In the next section, we explored the
interactions of typability and RSA, to present a more nuanced view of
typing's effect on valence ratings.

\subsection{Interactions}\label{interactions}

\begin{table}[tbp]
\begin{center}
\begin{threeparttable}
\caption{\label{tab:model-table1}Area under curve model statistics}
\begin{tabular}{lccccccc}
\toprule
Word Type & Model & $df$ & AIC & BIC & $\chi^2$ & $\Delta\chi^2$ & $p$\\
\midrule
Pseudo & Intercept Only & 2 & 130416.47 & 130433.25 & -65206.24 & NA & NA\\
Pseudo & Random Intercept & 3 & 122403.95 & 122429.12 & -61198.97 & 8014.52 & < .001\\
Pseudo & Adjustor Variable & 4 & 122405.44 & 122439.00 & -61198.72 & 0.51 & .476\\
Pseudo & Main Effects & 8 & 122204.31 & 122271.44 & -61094.16 & 209.13 & < .001\\
Pseudo & Interactions & 19 & 122197.68 & 122357.09 & -61079.84 & 28.64 & .003\\
Real & Intercept Only & 2 & 151926.51 & 151943.30 & -75961.26 & NA & NA\\
Real & Random Intercept & 3 & 150478.20 & 150503.38 & -75236.10 & 1450.32 & < .001\\
Real & Adjustor Variable & 4 & 150457.14 & 150490.72 & -75224.57 & 23.06 & < .001\\
Real & Main Effects & 8 & 150354.18 & 150421.34 & -75169.09 & 110.96 & < .001\\
Real & Interactions & 19 & 150315.00 & 150474.50 & -75138.50 & 61.18 & < .001\\
\bottomrule
\addlinespace
\end{tabular}
\begin{tablenotes}[para]
\textit{Note.} AIC: Aikaike Information Criterion, BIC: Bayesian Information Criterion
\end{tablenotes}
\end{threeparttable}
\end{center}
\end{table}

Next, the four-way interaction of typing speed, finger switching, hand
switching, and RSA was entered into the equation, including all the
smaller two- and three-way interactions. We focused on the most complext
interaction found, breaking down interaction terms into simple slopes of
low (-1SD), average, and high (+1SD) to explore each effect. For
example, if the four-way interaction was significant, one variable would
be broken into simple slopes, and the next most complex interactions
would be examined. This procedure was iterated until the interactions
were no longer significant or only main effects were examined. When
multiple interactions were present, we choose a common variable to help
break down the interactions with the least number of steps. Table
\ref{tab:model-table1} portrays that the addition of the interaction
components was significant for both pseudoword, \emph{p} = .003, and
real word, \emph{p} \textless{} .001, models.

\subsubsection{Pseudoword Simple Slopes}\label{pseudoword-simple-slopes}

For pseudowords, finger switches X RSA, \emph{b} = 0.02, \emph{t}(31984)
= 3.49, \emph{p} \textless{} .001, and typing speed X RSA, \emph{b} =
0.00, \emph{t}(31984) = -2.15, \emph{p} = .031 were the only significant
interactions.

HERE I THINK WE SHOULD BREAK THIS DOWN BY RSA SINCE IT'S THE SIMILAR ONE
BETWEEN THE TWO.

\subsubsection{Real Word Simple Slopes}\label{real-word-simple-slopes}

For real words, the three-way interactions of finger switch X hand
switch X RSA, \emph{b} = -0.01, \emph{t}(32130) = -5.88, \emph{p}
\textless{} .001, and speed X finger switch X hand switch, \emph{b} =
0.00, \emph{t}(32130) = -2.64, \emph{p} = .008, were the largest
significant interaction predictors.

BREAK THIS DOWN BY FINGER, THEN HAND, SEE WHAT HAPPENS WITH SPEED AND
RSA

All interaction statistics are included online in an Excel sheet at our
OSF page.

\section{Experiment 2}\label{experiment-2}

\section{Method}\label{method-1}

\subsection{Participants}\label{participants-1}

Similar to Experiment 1, 60 participants were recruited from the
university undergraduate human subject pool and received course credit
for their time. 7200 rows of data were present for these participants,
and no data was missing. Rating data were screened for multivariate
outliers. Again, part of one participant's ratings were found to have
extreme Mahalanobis distance scores ({\textbf{???}}). However, this
individual's ratings were left in the data set. Approximately 8.3
percent of the sample was left-handed. The average typing speed was
45.12 (\emph{SD} = 11.65), and the average percent accuracy rate for the
typing test was 93.58 (\emph{SD} = 5.31).

\subsection{Materials}\label{materials-1}

In this experiment, a smaller subset of words (120) from Experiment 1
were used, which were split evenly between pseudowords and real words.
Average word length was 3.80 (\emph{SD} = 0.40; range = 3 - 4).

\subsection{Procedure}\label{procedure-1}

In this study, when participants were shown the word (or pseudoword) on
the screen, they were first asked to type the word on the keyboard in
front of them. After they had typed the word, they were then asked to
rate the word for pleasantness using the scale and self-assessment
manikin discussed previously.

\section{Results}\label{results-1}

\subsection{Main Effects}\label{main-effects-1}

\begin{table}[tbp]
\begin{center}
\begin{threeparttable}
\caption{\label{tab:model-table2}Area under curve model statistics}
\begin{tabular}{lccccccc}
\toprule
Word Type & Model & $df$ & AIC & BIC & $\chi^2$ & $\Delta\chi^2$ & $p$\\
\midrule
Pseudo & Intercept Only & 2 & 13715.38 & 13727.76 & -6855.69 & NA & NA\\
Pseudo & Random Intercept & 3 & 12190.54 & 12209.10 & -6092.27 & 1526.85 & < .001\\
Pseudo & Adjustor Variable & 4 & 12192.17 & 12216.92 & -6092.08 & 0.37 & .543\\
Pseudo & Main Effects & 8 & 12152.25 & 12201.76 & -6068.13 & 47.91 & < .001\\
Pseudo & Interactions & 19 & 12127.85 & 12245.44 & -6044.93 & 46.40 & < .001\\
Real & Intercept Only & 2 & 16133.38 & 16145.76 & -8064.69 & NA & NA\\
Real & Random Intercept & 3 & 15922.15 & 15940.72 & -7958.08 & 213.23 & < .001\\
Real & Adjustor Variable & 4 & 15898.69 & 15923.45 & -7945.35 & 25.46 & < .001\\
Real & Main Effects & 8 & 15812.18 & 15861.69 & -7898.09 & 94.51 & < .001\\
Real & Interactions & 19 & 15734.48 & 15852.06 & -7848.24 & 99.71 & < .001\\
\bottomrule
\addlinespace
\end{tabular}
\begin{tablenotes}[para]
\textit{Note.} AIC: Aikaike Information Criterion, BIC: Bayesian Information Criterion
\end{tablenotes}
\end{threeparttable}
\end{center}
\end{table}

\subsection{Interactions}\label{interactions-1}

\section{Discussion}\label{discussion}

YADA SCHMADA CHANGE THIS SECTION These results imply that the QWERTY
keyboard has influenced our perceptions of words, in a more complex way
than a simple body specificity hypothesis. In the overall normed
database analyses, the original QWERTY effect was replicable across a
large body of various types of stimuli (verbs, Twitter, category norms),
with much the same size of effect as Jasmin and Casasanto (2012)
published. Word length was often negatively related to valence ratings,
which indicated that we like shorter words to type. Average letter
frequency was usually a positive predictor of valence ratings wherein
ratings are higher for words with more frequent letters; however, these
effects were inconsistent. Our measure of fluency (switches) varied
across stimulus sets but it appears, by analyzing multiple sources of
ratings for words at the same time, that there might have been an
interaction between RSA and number of switches. This interaction
portrayed that we find words that switch off of left-handed keypresses
as more pleasant, while right-handed keypresses are preferable by
switching hands less often. These effects were examined in more detail
in Experiment 2, which incorporated Beilock and Holt (2007) study by
including typing speed as a measure of expertise. Word ratings turned
out to be quite complex with a four-way interaction between
real/pseudowords, switches, RSA, and typing speed. All analyses showed a
positive effect of right-side words, as well as if they were shorter and
used more frequent letters. However, for pseudowords, no other effects
were significant. Both Beilock and Holt (2007) and Van den Bergh et al.
(1990) showed expert preferences for two and three letter combinations
that were typed with different fingers. Our results could imply that our
embodied actions influence preferences for procedures that are more
likely in our environment. While our pseudowords were legal English
phoneme combinations, they are extremely unlikely to have been
previously practiced or encountered in our daily tasks. Therefore,
switching preference will not extend to pseudowords (unpracticed
actions) because they are not fluent (Oppenheimer, 2008).

The effect of expertise was shown on real words, where the three-way
interaction between RSA, switches, and typing speed was examined by
separating out right, equal, and left-handed words. For right-handed
words, typing speed (or the interaction) was not a significant predictor
of valence, and while not significant, number of switches was negatively
related to valence ratings. For equally right-left and left-handed
words, pleasantness ratings increase by switching back and forth to the
right hand. Further, left-handed words showed an interaction between our
two embodied cognition variables, where the number of switches increases
valence ratings as the typing speed of the participant decreases.
Therefore, it appears that as participants gain fluency through
increased typing speed, the number of switches back and forth for
left-handed words matters less for pleasantness ratings. Many of the
most frequent letters on the QWERTY keyboard are on the left side, which
may frustrate a slow typist because of the need to coordinate finger
press schemata that involve same finger muscle movements (Rumelhart \&
Norman, 1982). Consequently, the number of switches becomes increasingly
important to help decrease interference from the need to continue to use
the same hand. The ease of action by switching back and forth is then
translated as positive feelings for those fluent actions (Oppenheimer,
2008).

These embodied results mirror a clever set of studies by Holt and
Beilock (2006) wherein they showed participants sentences that matched
or did not match a set of pictures (i.e.,the umbrella is in the air
paired with a picture of an open umbrella). Given dual-coding theory
({\textbf{???}}), it was not surprising that participants were faster to
indicate picture-sentence matches than non-matches (also see
{\textbf{???}}, {\textbf{???}}). Further, they showed these results
extended to an expertise match; hockey and football players were much
faster for sentence-picture combinations that matched within their sport
than non-matches, while novices showed no difference in speed for
matches or non-matches on sports questions. Even more compelling are
results that these effects extend to fans of a sport and are consistent
neurologically (i.e.,motor cortex activation in experts;
{\textbf{???}}). These studies clearly reinforce the idea that expertise
and fluency unconsciously affect our choices, even when it comes to
perceived pleasantness of words.

This extension of the QWERTY effect illuminates the need to examine how
skill can influence cognitive processes. Additionally, typing style,
while not recorded in this experiment, could potentially illuminate
differences in ratings across left-handed and right-handed words.
Hunt-and-peck typists are often slower than the strict typing manual
typists, which may eliminate or change the effects of RSA and switches
since typists may not follow left or right hand rules and just switch
hands back and forth regardless of key position. The middle of a QWERTY
layout also poses interesting problems, as many typists admit to
\enquote{cheating} the middle letters, such as t, and y or not even
knowing which finger should actually type the b key. Further work could
also investigate these effects on other keyboard layouts, such as
Dvorak, which was designed to predominately type by alternating hands to
increase speed and efficiency ({\textbf{???}}).

\newpage

\section{References}\label{references}

\setlength{\parindent}{-0.5in} \setlength{\leftskip}{0.5in}

\hypertarget{refs}{}
\hypertarget{ref-Barsalou1999}{}
Barsalou, L. W. (1999). Perceptual symbol systems. \emph{Behavioral and
Brain Sciences}, \emph{22}(4), 577--660.
doi:\href{https://doi.org/10.1017/S0140525X99002149}{10.1017/S0140525X99002149}

\hypertarget{ref-Beilock2007}{}
Beilock, S. L., \& Holt, L. E. (2007). Embodied Preference Judgments.
\emph{Psychological Science}, \emph{18}(1), 51--57.
doi:\href{https://doi.org/10.1111/j.1467-9280.2007.01848.x}{10.1111/j.1467-9280.2007.01848.x}

\hypertarget{ref-Cartmill2012}{}
Cartmill, E., Goldin-Meadow, S., \& Beilock, S. L. (2012). A word in the
hand: Human gesture links representations to actions.
\emph{Philosophical Transactions of the Royal Society B: Biological
Sciences}, \emph{367}, 129--143.

\hypertarget{ref-Casasanto2009}{}
Casasanto, D. (2009). Embodiment of abstract concepts: Good and bad in
right- and left-handers. \emph{Journal of Experimental Psychology:
General}, \emph{138}(3), 351--367.
doi:\href{https://doi.org/10.1037/a0015854}{10.1037/a0015854}

\hypertarget{ref-Casasanto2011}{}
Casasanto, D. (2011). Different Bodies, Different Minds. \emph{Current
Directions in Psychological Science}, \emph{20}(6), 378--383.
doi:\href{https://doi.org/10.1177/0963721411422058}{10.1177/0963721411422058}

\hypertarget{ref-Davidson1992}{}
Davidson, R. J. (1992). Anterior cerebral asymmetry and the nature of
emotion. \emph{Brain and Cognition}, \emph{20}(1), 125--151.
doi:\href{https://doi.org/10.1016/0278-2626(92)90065-T}{10.1016/0278-2626(92)90065-T}

\hypertarget{ref-Glenberg2009}{}
Glenberg, A. M., Webster, B. J., Mouilso, E., Havas, D., \& Lindeman, L.
M. (2009). Gender, emotion, and the embodiment of language
comprehension. \emph{Emotion Review}, \emph{1}(2), 151--161.
doi:\href{https://doi.org/10.1177/1754073908100440}{10.1177/1754073908100440}

\hypertarget{ref-Hauk2004}{}
Hauk, O., Johnsrude, I., \& Pulvermüller, F. (2004). Somatotopic
representation of action words in human motor and premotor cortex.
\emph{Neuron}, \emph{41}(2), 301--307.
doi:\href{https://doi.org/10.1016/S0896-6273(03)00838-9}{10.1016/S0896-6273(03)00838-9}

\hypertarget{ref-Havas2007}{}
Havas, D. A., Glenberg, A. M., \& Rinck, M. (2007). Emotion simulation
during language comprehension. \emph{Psychonomic Bulletin \& Review},
\emph{14}(3), 436--441.
doi:\href{https://doi.org/10.3758/BF03194085}{10.3758/BF03194085}

\hypertarget{ref-Holt2006}{}
Holt, L. E., \& Beilock, S. L. (2006). Expertise and its embodiment:
Examining the impact of sensorimotor skill expertise on the
representation of action-related text. \emph{Psychonomic Bulletin \&
Review}, \emph{13}(4), 694--701.
doi:\href{https://doi.org/10.3758/BF03193983}{10.3758/BF03193983}

\hypertarget{ref-Hommel2001}{}
Hommel, B., Müsseler, J., Aschersleben, G., \& Prinz, W. (2001). The
Theory of Event Coding (TEC): A framework for perception and action
planning. \emph{Behavioral and Brain Sciences}, \emph{24}(05), 849--878.
doi:\href{https://doi.org/10.1017/S0140525X01000103}{10.1017/S0140525X01000103}

\hypertarget{ref-Inhoff1997}{}
Inhoff, A. W., \& Gordon, A. M. (1997). Eye Movements and Eye-Hand
Coordination During Typing. \emph{Current Directions in Psychological
Science}, \emph{6}(6), 153--157.
doi:\href{https://doi.org/10.1111/1467-8721.ep10772929}{10.1111/1467-8721.ep10772929}

\hypertarget{ref-Jasmin2012}{}
Jasmin, K., \& Casasanto, D. (2012). The QWERTY Effect: How typing
shapes the meanings of words. \emph{Psychonomic Bulletin \& Review},
\emph{19}(3), 499--504.
doi:\href{https://doi.org/10.3758/s13423-012-0229-7}{10.3758/s13423-012-0229-7}

\hypertarget{ref-Logan1999}{}
Logan, F. A. (1999). Errors in copy typewriting. \emph{Journal of
Experimental Psychology: Human Perception and Performance},
\emph{25}(6), 1760--1773.
doi:\href{https://doi.org/10.1037//0096-1523.25.6.1760}{10.1037//0096-1523.25.6.1760}

\hypertarget{ref-Logan2003}{}
Logan, G. D. (2003). Simon-type effects: Chronometric evidence for
keypress schemata in typewriting. \emph{Journal of Experimental
Psychology: Human Perception and Performance}, \emph{29}(4), 741--757.
doi:\href{https://doi.org/10.1037/0096-1523.29.4.741}{10.1037/0096-1523.29.4.741}

\hypertarget{ref-Logan1998}{}
Logan, G. D., \& Zbrodoff, N. J. (1998). Stroop-type interference:
Congruity effects in color naming with typewritten responses.
\emph{Journal of Experimental Psychology: Human Perception and
Performance}, \emph{24}(3), 978--992.
doi:\href{https://doi.org/10.1037/0096-1523.24.3.978}{10.1037/0096-1523.24.3.978}

\hypertarget{ref-Lyons2010}{}
Lyons, I. M., Mattarella-Micke, A., Cieslak, M., Nusbaum, H. C., Small,
S. L., \& Beilock, S. L. (2010). The role of personal experience in the
neural processing of action-related language. \emph{Brain and Language},
\emph{112}(3), 214--222.
doi:\href{https://doi.org/10.1016/j.bandl.2009.05.006}{10.1016/j.bandl.2009.05.006}

\hypertarget{ref-Newell1976}{}
Newell, A., \& Simon, H. A. (1976). Computer science as empirical
inquiry: symbols and search. \emph{Communications of the ACM},
\emph{19}(3), 113--126.
doi:\href{https://doi.org/10.1145/360018.360022}{10.1145/360018.360022}

\hypertarget{ref-Oppenheimer2008}{}
Oppenheimer, D. M. (2008). The secret life of fluency. \emph{Trends in
Cognitive Sciences}, \emph{12}(6), 237--241.
doi:\href{https://doi.org/10.1016/j.tics.2008.02.014}{10.1016/j.tics.2008.02.014}

\hypertarget{ref-Ping2009}{}
Ping, R. M., Dhillon, S., \& Beilock, S. L. (2009). Reach for what you
like: The body's role in shaping preferences. \emph{Emotion Review},
\emph{1}(2), 140--150.
doi:\href{https://doi.org/10.1177/1754073908100439}{10.1177/1754073908100439}

\hypertarget{ref-Rieger2004}{}
Rieger, M. (2004). Automatic keypress activation in skilled typing.
\emph{Journal of Experimental Psychology: Human Perception and
Performance}, \emph{30}(3), 555--565.
doi:\href{https://doi.org/10.1037/0096-1523.30.3.555}{10.1037/0096-1523.30.3.555}

\hypertarget{ref-Rumelhart1982}{}
Rumelhart, D., \& Norman, D. (1982). Simulating a skilled typist: a
study of skilled cognitive-motor performance. \emph{Cognitive Science},
\emph{6}(1), 1--36.
doi:\href{https://doi.org/10.1016/S0364-0213(82)80004-9}{10.1016/S0364-0213(82)80004-9}

\hypertarget{ref-Salthouse1986}{}
Salthouse, T. A. (1986). Perceptual, cognitive, and motoric aspects of
transcription typing. \emph{Psychological Bulletin}, \emph{99}(3),
303--319.
doi:\href{https://doi.org/10.1037/0033-2909.99.3.303}{10.1037/0033-2909.99.3.303}

\hypertarget{ref-Simon1990}{}
Simon, J. R. (1990). The effects of an irrelevant directional cue on
human information processing. In R. Proctor \& T. Reeve (Eds.),
\emph{Stimulus--response compatibility: An integrated perspective} (pp.
31--86). Amsterdam.

\hypertarget{ref-Simon1969}{}
Simon, J. R., \& Small, A. M. (1969). Processing auditory information:
Interference from an irrelevant cue. \emph{Journal of Applied
Psychology}, \emph{53}(5), 433--435.
doi:\href{https://doi.org/10.1037/h0028034}{10.1037/h0028034}

\hypertarget{ref-Tettamanti2005}{}
Tettamanti, M., Buccino, G., Saccuman, M. C., Gallese, V., Danna, M.,
Scifo, P., \ldots{} Perani, D. (2005). Listening to action-related
sentences activates fronto-parietal motor circuits. \emph{Journal of
Cognitive Neuroscience}, \emph{17}(2), 273--281.
doi:\href{https://doi.org/10.1162/0898929053124965}{10.1162/0898929053124965}

\hypertarget{ref-VandenBergh1990}{}
Van den Bergh, O., Vrana, S., \& Eelen, P. (1990). Letters from the
heart: Affective categorization of letter combinations in typists and
nontypists. \emph{Journal of Experimental Psychology: Learning, Memory,
and Cognition}, \emph{16}(6), 1153--1161.
doi:\href{https://doi.org/10.1037/0278-7393.16.6.1153}{10.1037/0278-7393.16.6.1153}

\hypertarget{ref-Yang2009}{}
Yang, S.-J., Gallo, D. A., \& Beilock, S. L. (2009). Embodied memory
judgments: A case of motor fluency. \emph{Journal of Experimental
Psychology: Learning, Memory, and Cognition}, \emph{35}(5), 1359--1365.
doi:\href{https://doi.org/10.1037/a0016547}{10.1037/a0016547}

\hypertarget{ref-Zwaan1999}{}
Zwaan, R. A. (1999). Embodied cognition, perceptual symbols, and
situation models. \emph{Discourse Processes}, \emph{28}(1), 81--88.
doi:\href{https://doi.org/10.1080/01638539909545070}{10.1080/01638539909545070}

\hypertarget{ref-Zwaan2006}{}
Zwaan, R. A., \& Taylor, L. J. (2006). Seeing, Acting, Understanding:
Motor Resonance in Language Comprehension. \emph{Journal of Experimental
Psychology: General}, \emph{135}(1), 1--11.
doi:\href{https://doi.org/10.1037/0096-3445.135.1.1}{10.1037/0096-3445.135.1.1}






\end{document}
